\section{Boundary Conditions}
\label{sec:BoundaryConditions}

When using our proposed system, various boundary conditions regarding integration and execution must be taken into account. It is designed to be integrated into an existing CI/CD pipeline of a client application but can also be run locally via a CLI. When it is used as a step in an existing CI/CD pipeline, it must be part of a stage prior to releasing in order to ensure that the most recent version of the Web API is used. Client developers must specify multiple configuration parameters to setup the system. While configuring a migration strategy and specifying a target programming language are mandatory tasks, users can optionally define a linting rule set which determines the code formatting of the emitted code. 

On its initial run, our proposed system does not require a migration guide as the latest version of the Web API is used as a starting point for generating the persistent facade. Hence, only the \ac{URI} of the IDL document needs to be present. After that, each run requires specifying the \acp{URI} of both, the IDL document and the migration guide in order to adapt the facade and update the library subsystems. Omitting mandatory parameters results in an error message during runtime that highlights the missing configuration. Optional parameters do not trigger error messages as a default configuration is used instead. Detecting missing configuration parameters and specifying error messages are tasks of the \texttt{Executable} subsystem.

In addition to configuration parameters, unmet preconditions also raise errors. If the system fails to fetch either the IDL document or the migration guide, no output can be generated and execution will be aborted. The specific reason for the failure is given in the CI / CD system's log or on the command line. Furthermore, if changes specified in the migration guide are incompatible or inconsistent, the facade adaptation will fail and an error message will be logged. Removing a parameter from a non-parameterized method or adding and removing the same parameter to and from a method at the same time are examples of incompatible or inconsistent changes, respectively. Errors related to fetching and parsing of an IDL document or migration guide are managed by the \texttt{IDL} \texttt{Importer} subsystem and \texttt{Migration} \texttt{Guide} \texttt{Importer} subsystem. Checking for incompatibilities and inconsistencies within the migration guide is done by the \texttt{Migrator} component.



