\subsection{Semi-Automated Migration}\label{subsec:EvalSemiAutomated}
Changes categorized as semi-automatable can be migrated by our proposed system, but the adapted client library is incapable of reproduce the previous state exactly. Nonetheless, our proposed system provides additional support for client developers to manually migrate their application. Changes in this category are reflected in the IDL document of the Web APIs either in modifications to their elements or in their absence.


\begin{description}
	\item[Category:] \textsc{Semi-Automated Migration} \newline Changes of this type can be further divided into changes that can be migrated but generate an error at runtime if the client library is used incorrectly and changes that can be migrated but do not restore the original functionality. New approaches must be explored for the automatic migration of the latter.
	\item[Problem:] Changes in this category require client developers to manually migrate their application in addition to the automated refactoring performed by our proposed system. Without manual adjustments, these changes lead to runtime errors. A full list of these changes is shown below:
	\begin{itemize}
		\item \textit{Removing changes} 
		\item \textit{Parameter boundaries}
		\item \textit{Requiring authentication}
		\item \textit{Multi-version migrations}
	\end{itemize}
	\item[Solution:] The problems are reflected in the IDL document, which enables a semi-automated migration. In addition, removing changes can be stated in the machine-readable migration guide, whereby fallback values can be specified for certain targets that enable their fully automated migration. Targets that do not allow specifying a fallback value are provided with the \texttt{@available} annotation, which returns an error message stating the reason for the removal at compile time. Changes to the parameter boundaries are automatically adjusted at the library level, so that client developers receive an error message if they are exceeded. All methods generated by \textsc{Pallidor} enable the provision of authentication information which can be stored centrally in the \texttt{NetworkManager} struct. In the event that a method's authentication requirement changes, this information is automatically used, implied that it has been specified once by the client developers. For migrating the client library between multiple versions of a Web API, future research should be conducted to automatically generate a migration guide on demand for a particular version. 
	\item[Benefits:] Although the changes cannot be automatically migrated, our proposed system provides helpful support for client developers. Errors caused by changes in this category can only be identified at runtime, when a Web API \newpage is manually integrated. By using the generated client library, which directly reflects these changes, the detection of these errors is shifted to compile time. In addition, by migrating removed elements, we integrate detailed error messages that are displayed as compilation errors by Xcode. Web API providers can use them to explain their clients why the affected element was removed and, if possible, what further steps can be taken. Compliance with parameter boundaries is ensured by preconditions, which are automatically adjusted whenever the boundaries change. Unit tests should be defined by client developers so that they are informed about these changes and can make appropriate adjustments before they publish a new version of their application. The design of the client library allows its users to avoid the problem of authentication requirements that are added after the initial integration of the Web API. 
	\item[Consequences:] Providing error messages for removed elements does not restore the Web API dependency to its original state. Replacing the removed functionality must be done manually by client developers. Since unit tests have to be created manually by client developers, a change in the parameter boundaries remains unnoticed if they are omitted. The use of central authentication information is only possible if it is already used elsewhere in the client application. If the Web API did not previously require any authentication or if no such method was used, there is no information that could be automatically added. Until they can be generated, providers must manually create and maintain migration guides for each version of their Web API.
	\item[Tested By:] While unit tests are used to validate the migration of removed elements, changes in authentication requirements and parameter boundaries are tested by scripts that can be locally executed from the root directory of the \textsc{Pallidor} Swift package. They require to specify the target directory of the generated Swift files using the \texttt{-t} parameter. Each script generates separate Swift packages for the unchanged and updated version of the Web API to facilitate the comparison of the changed elements. Script files are highlighted in \textit{italic} while unit tests are prefixed with the \texttt{test} keyword. For multi-version migrations, example test cases are specified in the \texttt{.md} file that describes the process of how migration guides from several versions can be merged into one migration guide for a specific version step. These test cases can be used by future research to integrate them in the automatic generation of migration guides.
\end{description}
\vspace{0.25cm}
\textbf{Removing} elements of a Web API without providing a replacement cannot be migrated to restore their original state. In order to support client developers, \textsc{Pallidor} annotates removed elements with Swift's \texttt{@available} attribute. So whenever a removed item is accessed, a compiler error is displayed explaining the reason for the removal. All tests for migrating removed elements are listed in Table \ref{tab:RemoveMigrationTests}.

	\begin{center}
		\begin{longtable}{@{}lp{0.18\textwidth}lp{0.4\textwidth}@{}}
			\toprule
			\textbf{Target} & \textbf{Test} & \textbf{Identifier} & \textbf{Description} \\ \midrule \endhead
			Model           &   test\-Deleted\-Model   &   \texttt{ApiResponse}      &  The model \texttt{ApiResponse} is removed from the schema components of the Web API.      \\
			Attribute       &  test\-Deleted\-Property    &  \texttt{weight}   &   The attribute is removed from the \texttt{Pet} model. A fallback value is provided to prevent the client application from breaking.       \\
			Endpoint        &  test\-Deleted    &    \texttt{/pet}     &   The endpoint \texttt{Pet} is removed from the Web API. This includes removing all of its methods. \\
			Method        &  test\-Deleted\-Method       &    \texttt{addPet}    &       The method is removed from the \texttt{Pet} endpoint.                          \\
			Parameter       &  test\-Deleted\-Parameter      &    \texttt{username}       &   The parameter is removed from the \texttt{updateUser} method of the \texttt{User} endpoint. A fallback value is provided to prevent the client application from breaking.        \\ 
			 Return Value    &            \multicolumn{1}{c}{-}  &    \multicolumn{1}{c}{-}                   &      Methods must specify a return value. Therefore, removing changes cannot target return values.    \\ \bottomrule
					\caption{Remove migrations for all target types}
			\label{tab:RemoveMigrationTests}
		\end{longtable}
	\end{center}
\vspace{-1cm}
Web API providers can specify fallback values for \texttt{Parameters} and \texttt{Attributes} that are incorporated by \textsc{Pallidor} to prevent the client application from breaking. Each method must specify a type that is returned upon successful completion of the request. Hence, removing changes cannot target return values. 

The remaining change types in this category modify the semantics of the Web API. Our proposed system provides a means for client developers to automatically migrate these changes, provided that certain requirements are met. Tests for these change types are listed in Table \ref{tab:OtherSemiAutomatedChangeTypesTests}.

\textbf{Changing the boundaries of parameters} affect the preconditions of invoking a method of a Web API. Supplying a value that falls below the lower or exceeds the upper boundary,  results in an error message issued by the Web API. Extending boundaries does not break client applications as the existing implementation is still valid. However, narrowing boundaries can lead to the situation that the previously valid values are no longer within the limits.
\newpage
\textbf{Requiring authentication} of previously unauthorized functionality forces client developers to authenticate before requesting services. If no authorization information is transmitted, the request fails and an error message is issued by the Web API.

\textbf{Multi-version migrations} are currently not supported by \textsc{Pallidor}. It supports migrating client libraries between two versions of a Web API. Therefore, their providers must specify and maintain a migration guide for every version ever released.

	\begin{center}
	\begin{longtable}{@{}>{\raggedright\arraybackslash}p{0.19\textwidth}p{0.26\textwidth}p{0.5\textwidth}@{}}
		\toprule
		\textbf{Change} & \textbf{Test}  & \textbf{Description} \\ \midrule \endhead
		 Parameter boundaries &          \textit{changeBoundaries.sh}               &      The parameter \texttt{orderId} of the \texttt{getOrderById} method changed its boundaries from accepting values in the range of 1-9999 to values in the range of 1000-4999.         \\ 
		Requiring authentication &    \textit{changeAuth.sh}          &         The method \texttt{placeOrder} now requires users to be authenticated while the method \texttt{updatePet} is publicly accessible without authenticating. Their \texttt{authorization} parameters changed from optional to non-optional and vice versa.    \\
		Multi-version migration & \multicolumn{1}{c}{-}  & Currently unsupported. Further investigation is required to develop an algorithm to generate a machine-readable migration guide for a specific version step.  \\
		\bottomrule
		\caption{Further tests of semi-automatable change types}
		\label{tab:OtherSemiAutomatedChangeTypesTests}	
	\end{longtable}
\end{center}
\vspace{-1cm}


Changing the boundaries of parameters and requesting authentication of a previously unauthenticated method are tested by generating a Swift library for each version of the Web API. In order to support the development of multi-version migrations in the future, we provide an abstract approach on merging multiple migration guides in \texttt{MultiVersionMigration.md}. It provides a brief overview of an examplatory approach to combine multiple migration guides of various versions of a method into a single migration guide for migrating the method from its original version to the latest version available.
