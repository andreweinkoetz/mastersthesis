\section{Implementation}
\label{sec:Implementation}

For evaluating the proposed system, we created \textit{Pallidor} a prototypical implementation using the Swift programming language. It is composed of multiple subsystems that are developed and maintained as standalone Swift packages. In addition to our own implementations, third-party components are integrated using the Swift Package Manager. All packages are published under an open-source license. Their source code is publicly available in their respective repositories on Github. Pallidor is composed of three Swift packages that are combined in an executable that provides a commandline user interface. The \textit{PallidorGenerator} package implements the subsystems \texttt{IDL Importer} and \texttt{IDL Conversion}. The \textit{PallidorMigrator} package integrates the functionality of the \texttt{Migration Manager}, \texttt{Migration Guide Importer} and \texttt{SourceCode Importer} subsystems. The \textit{Pallidor} package contains the commandline user interface and the \texttt{Persisting Manager} subsystem. For the individual components, we identified several open-source Swift packages that provide the required functionality.

\renewcommand{\arraystretch}{1.4}
\begin{table*}[ht]
	\begin{center}
		\begin{tabular}{|>{\centering\arraybackslash}m{2.7cm}|>{\centering\arraybackslash}m{3cm}|>{\centering\arraybackslash}m{3.2cm}|>{\centering\arraybackslash}m{4cm}|}
			\hline
			\begin{center}
				\textbf{Subsystem/ Component}
			\end{center} &  \begin{center}
				\textbf{Swift Package Name} 
			\end{center}&  \begin{center}
				\textbf{Swift Package Version}
			\end{center} &
		 \begin{center}
			\textbf{Swift Package Author / Publisher}
		\end{center} \\ \hline
			\textit{IDL Importer} & OpenAPIKit & 2.0.0 \newline (Sep. 28th, 2020) &
			Mathew Polzin \\ \hline
			\textit{Source Code Importer} & Sourcery & 1.0.0 \newline (Aug. 14th, 2020) &
			Krzysztof Zabłocki \\ \hline
			\textit{Code Formatter} & swift-format & 0.50300.0 \newline (Sep. 19th, 2020) &
			Apple, Inc. \\ \hline
		\end{tabular}
		\caption{Third-party Swift packages used for subsystems in Pallidor}\label{tbl:PallidorDep}
	\end{center}
\end{table*}

Since Pallidor is a prototype, its functionality is limited to generating a persistent Swift package from an OpenAPI specification. It can be integrated in client applications using Swift in version 5.2 or later. Since our generated Swift package uses Apple's \texttt{Combine}\footnote{https://developer.apple.com/documentation/combine} framework, the client application's execution environment must be at least iOS 13 or MacOS 10.15. 

\begin{itemize}
	\item OAI spec can not nest schemas
	\item other restrictions to specs (must define success return value), needs consistency with actual API (not string error defined but json msg returned)
	\item OAIGenerator drawbacks and why we cant use it?
\end{itemize}

