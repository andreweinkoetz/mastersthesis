\section{API Evolution Client Impact}
\label{sec:APIEvolutionClientImpact}

Client developers must devote a large extent of their efforts to keep their application compatible with an API in use. Research related to API evolution therefore also includes investigations of the impact on API consumers. Although there are studies related to the impact of changes in Web APIs, most of the studies examine applications using local libraries.

In \cite{brito_you_2020}, Brito et al. conducted a field study to determine the motivation of API providers to introduce breaking changes. They discovered that 39\% of these changes have a direct impact on consumers. The authors determined that although API providers state that the migration effort for their consumers is low (86\%) or moderate (14\%), 45\% of API-related questions were asked by client developers who have difficulties overcoming breaking changes. To find out about them, client developers need to examine a variety of documentation options as API developers plan to note their changes in release notes, change logs, source code, website, a migration guide, samples, or in a README file \cite{brito_you_2020}. In response to these challenges, API providers should focus on documenting changes in a single artifact to ease the effort of migrating for their consumers. Using a machine-readable migration manual ensures that all changes are documented in one artifact and can be easily found by consumers and tools.

Another large-scale study was conducted by Xavier et. al \cite{xavier_historical_2017}, analyzing the impact of API breaking changes in local libraries. They have seen the frequency of changes increase over time, which they associate with library development as they become more complex. Their findings show that 27.99\% of all API changes break backward compatibility. Although a median of only 2.54\% of clients are potentially affected by these changes, they have identified several outliers which show that in some cases every client application is affected. Based on their findings, they suggest using an impact analysis tool for library developers before changing commonly used APIs. While their study focuses on local libraries, the results apply to modern Web APIs as well. Furthermore, breaking backwards compatibility is a more serious problem for client developers because they cannot stick to an older version after a Web API has been updated.

Espinha et al. \cite{espinha_web_2014} further investigated the challenges for client developers regarding Web API changes. They found that the impact of changes largely depends on the quality of client application architectures. Hence, the authors prioritize separation of concerns in order to encapsulate components that are subject to change due to API development. According to their findings, the churn percentage for updating these well-designed client applications is 50\% lower than their average observations. This results in reduced efforts for maintaining an API usage, which requires at least 50\% of the initial setup effort or even exceeds it \cite{espinha_web_2014}. Hence, an automated migration process must consider the principle of separation of concerns in order to minimize the effort of integrating non-breaking changes and migrating breaking changes.