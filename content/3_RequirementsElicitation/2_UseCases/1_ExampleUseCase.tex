\subsubsection{Example Use Case}
\label{subsubsec:UseCase:ExampleUseCase}

\textit{Short general description of the use case. Use the table below as a template. Replace all CAPITAL LETTER PLACEHOLDERS with the content of the use case. The following list gives you some context to the different parts that describe a use case according to \cite{bass_software_2013}}
\textit{
    \begin{itemize}
        \item \textbf{Source / Source of stimulus}: The internal or external actor that generates the stimulus and how that actor is associated with the system.
        \item \textbf{Stimulus}: The event that arrives at the system. This can e.g., be an input, crash, or failure that requires a response from the system.
        \item \textbf{Environment}: The conditions such as preconditions and postconditions as well as invariants under which the stimulus requires a response. Examples for this can be normal operation, startup, shutdown, repair mode, or any previous clearly defined state.
        \item \textbf{Artefact}: The part of the system to which the use case applies. This might be the whole system or a subcomponent that reacts to the stimulus.
        \item \textbf{Response}: Specify how the system should respond to a stimulus. You can describe the response as a flow of events. These responses either consist of the responsibilities of the 
        \begin{itemize}
            \item system to perform a response for the stimulus (for runtime use cases) 
            \item developer to perform a response for the stimulus (for development-time use cases)
        \end{itemize}
        \item \textbf{Response Measure}: Measures that can be used to test the requirement and evaluate of your system satisfies this use case. You should evaluate this in the \nameref{ch:ObjectDesign} chapter using unit tests, system tests, and integration tests.
    \end{itemize}
}

\vspace{-2mm}
\begin{center}
    \def\arraystretch{1.5}
    \begin{longtable}{ p{0.22\linewidth} p{0.72\linewidth} }
    \hline
        \textit{Source} & \textsf{SOURCE}\\
    \hline
        \textit{Stimulus} & \textsf{STIMULUS}\\
    \hline
    	\textit{Environment} & \textsf{ENVIRONMENT}\\
    \hline
    	\textit{Artefact} & \textsf{ARTEFACT}\\
    \hline
    \textit{Response} &
    \vspace{-5.1mm}
    \begin{enumerate}[itemindent=-9pt, leftmargin=14pt, itemsep=0pt, align=left]
        \item FIRST EVENT
        \item SECOND EVENT
        \item THIRD EVENT
    \end{enumerate} \\ [-5mm]
    \hline
    \textit{Response Measure} &
    \vspace{-8.5mm}
    \begin{itemize}[itemindent=-9pt, leftmargin=14pt, itemsep=0pt, align=left]
       	\item RESPONSE MEASURE
       	\item RESPONSE MEASURE
        \vspace{-5mm}
    \end{itemize}\\
    \hline
    \end{longtable}
\end{center}