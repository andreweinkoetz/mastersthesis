%%%%%%%%%%%%%%%%%%%
% essentials
%%%%%%%%%%%%%%%%%%%

% use the scrbook KOMA document class
\documentclass[
		pdftex, 		%
		a4paper, 		% DIN A4 format
		titlepage,		% separate title page
		%draft,			% draft version, no figures in PDF!
		final,			% final version
		\printstyle,	% see \def in main document (oneside or twoside)
		12pt,			% font size
		DIV=calc,
		cleardoublepage=plain
]{scrbook}
						
\setkomafont{disposition}{\normalfont\bfseries}
\usepackage{scrhack}

% page geometry - define custom if needed...
\def\coverborderleft{20mm} % needed for title page - update: Druck mit Pappumschlag braucht's eher nicht
\usepackage[left=30mm , right=30mm, top=30mm, bottom=30mm]{geometry}

% typesetting
\usepackage{palatino} 				% Palatino font with sans-serif \sffamily Helvetica
\usepackage{acronym}
\usepackage[utf8]{inputenc} 			% Umlaute
\usepackage[T1]{fontenc}			% extended character set
\parindent0pt           			% no indentation of the first line
\parskip1ex             			% gap between paragraphs
\usepackage{setspace}				% package for line spacing settings
\singlespacing					% 1,0
%\onehalfspacing				% 1,5
%\doublespacing					% 2,0

% language for automated stuff, like "List of Figures" or "Abbildungsverzeichnis"
\usepackage[english]{babel}
%\usepackage[german=quotes]{csquotes} % Deutsche Anführungszeichen

% BibTex
\usepackage{natbib}
\bibliographystyle{alpha}



%%%%%%%%%%%%%%%%%%%
% figures & co
%%%%%%%%%%%%%%%%%%%

% for array tables
\usepackage{array}
\usepackage{multirow}
\newcommand{\tabitem}{~~\llap{\textbullet}~~}

% graphics and figures
\usepackage{graphicx, tikz, pgfplots}
\graphicspath{{images/}} % additional path(s) for loading images
\usepackage{rotating} % needed for \begin{sidewaysfigure} ... 

% insert PDF files
\usepackage{pdfpages}
% settings for PDF Pages to accept additonal versioned PDF files
\pdfminorversion=6
\pdfcompresslevel=9
\pdfobjcompresslevel=9

% define custom colors
\usepackage{xcolor}
\definecolor{gray1}{gray}{0.92}
\definecolor{darkgreen}{rgb}{0,0.5,0}
\definecolor{urlLinkColor}{rgb}{0,0,0.5}
\definecolor{LinkColor}{rgb}{0,0,0}
\definecolor{ListingBackground}{rgb}{0.85,0.85,0.85}

\usepackage{color}
\definecolor{LinkColor}{rgb}{0.1,0.1,0.1}
\definecolor{ListingBackground}{rgb}{0.98,0.98,0.98}
\definecolor{gray}{rgb}{0.4,0.4,0.4}
\definecolor{darkblue}{rgb}{0.0,0.0,0.6}
\definecolor{cyan}{rgb}{0.0,0.6,0.6}



%%%%%%%%%%%%%%%%%%%
% layout tools
%%%%%%%%%%%%%%%%%%%

% for \textblock on title page
\usepackage[absolute]{textpos}
%\setlength{\TPHorizModule}{1mm}
%\setlength{\TPVertModule}{\TPHorizModule}

% TUM Corporate Design definitions taken from official templates (tum.de/cd)
\newcommand{\UniversitaetLogoBreite}{19mm}
\newcommand{\UniversitaetLogoHoehe}{1cm}
\definecolor{UniversitaetFarbe}{RGB}{0,101,189}

% blindtext generator
\usepackage{lipsum}

% advanced conditionals
\usepackage{pdftexcmds}

\usepackage{longtable}


%%%%%%%%%%%%%%%%%%%
% math
%%%%%%%%%%%%%%%%%%%

\usepackage{amssymb}
\usepackage{amsmath}
% custom argmin, argmax
\newcommand{\argmax}[1]{\underset{#1}{\operatorname{arg}\,\operatorname{max}}\;}
\newcommand{\argmin}[1]{\underset{#1}{\operatorname{arg}\,\operatorname{min}}\;}



%%%%%%%%%%%%%%%%%%%
% code listings
%%%%%%%%%%%%%%%%%%%

\newcommand{\srcsize}{\@setfontsize{\srcsize}{5pt}{5pt}}

\usepackage{listings}
\lstloadlanguages{TeX, C++, XML, Matlab, Java, Python, C, Swift} % add languages if needed 
\lstset{%
	language=[LaTeX]TeX,     %
	numbers=left,            % line numbers left
	stepnumber=1,            % number every line
	numbersep=5pt,           % distance of line numbers to the code
	numberstyle=\tiny,       % size of the line numbers
	breaklines=true,         % break lines if necessary
	breakautoindent=true,    % indent lines after line breaks
	postbreak=\space,        % line break at spaces
	tabsize=2,               % size of tab indentation
	basicstyle={\ttfamily\small},
	showspaces=false,        % don't show spaces
	columns=fullflexible,
	showstringspaces=false,  % ...also not in 'strings' / "strings"
	extendedchars=true,      % show all Latin1 chars
	backgroundcolor=\color{ListingBackground} % background color of the listing
}

\lstdefinelanguage{swift}
{
	morekeywords={
		func,if,then,else,for,in,while,do,switch,case,default,where,break,continue,fallthrough,return,
		typealias,struct,class,enum,protocol,var,func,let,get,set,willSet,didSet,inout,init,deinit,extension,
		subscript,prefix,operator,infix,postfix,precedence,associativity,left,right,none,convenience,dynamic,
		final,lazy,mutating,nonmutating,optional,override,required,static,unowned,safe,weak,internal,
		private,public,is,as,self,unsafe,dynamicType,true,false,nil,Type,Protocol,
	},
	morecomment=[l]{//}, % l is for line comment
	morecomment=[s]{\\(}{)}, % s is for start and end delimiter
	morestring=[b]", % defines that strings are enclosed in double quotes
	keywordstyle=\color{keyword},
	stringstyle=\color{swiftstring},
	commentstyle=\color{comment}
}

\definecolor{keyword}{HTML}{BA2CA3}
\definecolor{swiftstring}{HTML}{D12F1B}
\definecolor{comment}{HTML}{008400}

\definecolor{delim}{RGB}{20,105,176}
\definecolor{numb}{RGB}{106, 109, 32}
\definecolor{string}{rgb}{0,0,0}

\lstdefinelanguage{json}{
	numbers=left,
	numberstyle=\tiny,
	rulecolor=\color{black},
	showspaces=false,
	breaklines=true,
	postbreak=\space,
	breakatwhitespace=true,
	basicstyle=\ttfamily\small,
	upquote=true,
	morestring=[b]",
	stringstyle=\color{string},
	literate=
	*{0}{{{\color{numb}0}}}{1}
	{1}{{{\color{numb}1}}}{1}
	{2}{{{\color{numb}2}}}{1}
	{3}{{{\color{numb}3}}}{1}
	{4}{{{\color{numb}4}}}{1}
	{5}{{{\color{numb}5}}}{1}
	{6}{{{\color{numb}6}}}{1}
	{7}{{{\color{numb}7}}}{1}
	{8}{{{\color{numb}8}}}{1}
	{9}{{{\color{numb}9}}}{1}
	{\{}{{{\color{delim}{\{}}}}{1}
	{\}}{{{\color{delim}{\}}}}}{1}
	{:}{{{\color{delim}{:}}}}{1}
	{[}{{{\color{delim}{[}}}}{1}
	{]}{{{\color{delim}{]}}}}{1},
}

% captions for listings
\usepackage{caption}
\DeclareCaptionFont{white}{\color{white}}
\DeclareCaptionFormat{listing}{\colorbox[cmyk]{0.43, 0.35, 0.35,0.01}{\parbox{\textwidth}{\hspace{15pt}#1#2#3}}}
\captionsetup[lstlisting]{format=listing,labelfont=white,textfont=white, singlelinecheck=false, margin=0pt, font={footnotesize}}



%%%%%%%%%%%%%%%%%%%
% header & footer
%%%%%%%%%%%%%%%%%%%

\usepackage{fancyhdr}
\pagestyle{fancy}
\fancyhead{}
\fancyfoot{} 
\renewcommand{\headrulewidth}{0.4pt} % line for header - 0.0pt = no line
\renewcommand{\footrulewidth}{0.0pt} % line for footer - 0.0pt = no line

\renewcommand{\chaptermark}[1]{\markboth{\thechapter\quad#1}{}}
\renewcommand{\sectionmark}[1]{\markright{\thesection\quad#1}}

\fancyhead[LO]{\textit{\rightmark}}	
\fancyhead[RO]{\textit{\thepage}}

\def\twoside{twoside} % macro needed
\ifx \printstyle \twoside
	% adapt footers and headers if two-sided
	\fancyhead[LE]{\textit{\thepage}}	
	\fancyhead[RE]{\textit{\rightmark}}
\fi



%%%%%%%%%%%%%%%%%%%
% hyperref
%%%%%%%%%%%%%%%%%%%

\usepackage[
	pdftitle={\titleFirstLanguage},
	pdfauthor={\authorname},
	pdfsubject={\titleFirstLanguage},
	pdfcreator={\authorname},
	pdfkeywords={\titleFirstLanguage, \authorname}, 
	pdfpagemode=UseOutlines,%                                  
	pdfdisplaydoctitle=true,%                                  
	pdflang=en%                                              
]{hyperref}

\hypersetup{%
	colorlinks=true,%        colored links without border
	linkcolor=LinkColor,%    
	citecolor=LinkColor,%    
	filecolor=LinkColor,%    
	menucolor=LinkColor,%    
	urlcolor=LinkColor,%     
	bookmarksnumbered=true%  
}

\usepackage{enumitem}

\makeatletter
\newcommand{\customlabel}[2]{%
   \protected@write \@auxout {}{\string \newlabel {#1}{{#2}{\thepage}{#2}{#1}{}} }%
   \hypertarget{#1}{}
}
\makeatother
